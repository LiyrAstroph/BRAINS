\documentclass[oneside]{book}
\usepackage{apjfonts}
\usepackage{geometry}
\usepackage{graphicx}

\DeclareMathAlphabet{\mathbi}{OT1}{cmr}{bx}{it}
\SetMathAlphabet\mathbi{bold}{OT1}{cmr}{bx}{it}

\begin{document}

\title{\bf BRAINS:\\
Bayesian Reverberation-mapping Analysis Integrated
with Nested Sampling}
\author{Yan-Rong Li\\
Institute of High Energy Physics}

\maketitle

\chapter{BLR modeling}

\section{Coordinate rotation}
\begin{figure}[h!]
\centering
\includegraphics[width=0.5\textwidth]{coord.pdf}
\end{figure}

Rotate an coordinate (XOY) by an angle of $\theta$ to (X'OY'), there are relations
\begin{equation}
\left[\begin{array}{c}
\mathbi{e}_{x'} \\
\mathbi{e}_{y'}
\end{array}\right]=\left[\begin{array}{cc}
\cos\theta & \sin \theta  \\
-\sin\theta &  \cos \theta       
\end{array}\right]\left[\begin{array}{c}
\mathbi{e}_{x} \\
\mathbi{e}_{y}
\end{array}\right].
\end{equation}
and 
\begin{equation}
\left[\begin{array}{c}
\mathbi{e}_{x} \\
\mathbi{e}_{y}
\end{array}\right]=\left[\begin{array}{cc}
\cos\theta & -\sin \theta  \\
\sin\theta &  \cos \theta       
\end{array}\right]\left[\begin{array}{c}
\mathbi{e}_{x'} \\
\mathbi{e}_{y'}
\end{array}\right].
\end{equation}
Therefore, for a vector $A$, its components in (XOY) and (X'OY') have relation 
\begin{equation}
A=[x', y'] \left[\begin{array}{c}
\mathbi{e}_{x'} \\
\mathbi{e}_{y'}
\end{array}\right] = [x, y]\left[\begin{array}{c}
\mathbi{e}_{x} \\
\mathbi{e}_{y}
\end{array}\right]=[x, y] \left[\begin{array}{cc}
\cos\theta & -\sin \theta  \\
\sin\theta &  \cos \theta       
\end{array}\right]\left[\begin{array}{c}
\mathbi{e}_{x'} \\
\mathbi{e}_{y'}
\end{array}\right].
\end{equation}
This yields
\begin{equation}
\left[\begin{array}{c}
x' \\
y'
\end{array}\right]=\left[\begin{array}{cc}
\cos\theta & \sin \theta  \\
-\sin\theta &  \cos \theta       
\end{array}\right]\left[\begin{array}{c}
x \\
y
\end{array}\right].
\end{equation}


In right-handed coordinate frame, we perform a rotation around $y$-axis by an angle of $l_\theta$ and 
then a rotation around $z$-axis by an angle of $l_\phi$. The transformation matrix is
\begin{equation}
\left[\begin{array}{ccc}
\cos l_\phi & \sin l_\phi & 0 \\
-\sin l_\phi &  \cos l_\phi & 0 \\
     0      &      0       & 1 
\end{array}\right]
\left[\begin{array}{ccc}
\cos l_\theta  & 0  & -\sin l_\theta\\
   0      &      1        &  0  \\
\sin l_\theta &  0  & \cos l_\theta 
\end{array}\right]=
\left[\begin{array}{ccc}
\cos l_\phi\cos l_\theta  & \sin l_\phi  & -\cos l_\phi\sin l_\theta\\
-\sin l_\phi\cos l_\theta      & \cos l_\phi        &  \sin l_\phi\sin l_\theta  \\
\sin l_\theta &  0  & \cos l_\theta 
\end{array}\right]
.
\end{equation}

\section{Time Lag}
The obsever is located at $(D\rightarrow\infty, 0, 0)$, i.e., the line of sight is along $x-$axis. 
For a cloud at $(x, y, z)$, its time lag is
\begin{equation}
\tau =\sqrt{x^2+y^2+z^2} + \sqrt{(D-x)^2+y^2+z^2} - D \approx r + D(1-x/D) - D \approx r - x,
\end{equation}
where $r=\sqrt{x^2+y^2+z^2}$.

The angle between the line of sight and the line of cloud is
\begin{equation}
 \cos \varphi = \frac{D\cdot x}{D\cdot r} = \frac{x}{r}.
\end{equation}


\end{document}

